\documentclass{article} % For LaTeX2e
\usepackage{nips12submit_e,times,url,comment,amsmath,amsfonts}
%\documentstyle[nips10submit_09,times,art10]{article} % For LaTeX 2.09

\usepackage{amsmath}
%\usepackage{amsthm}
\usepackage{amsfonts}
\usepackage{amssymb}
\newcommand{\RR}{\mathbb{R}}

\title{Cross-Lingual Document Retrieval through Hub Languages}

\begin{comment}
\author{
Jan Rupnik \\
A.I. Laboratory\\
Jo\v zef Stefan Institute\\
Ljubljana, Slovenia\\
\texttt{jan.rupnik@ijs.si} \\
\And
Andrej Muhi\v c\\
A.I. Laboratory\\
Jo\v zef Stefan Institute\\
Ljubljana, Slovenia\\
\texttt{andrej.muhic@ijs.si} \\
\AND
Primo\v z \v Skraba \\
A.I. Laboratory\\
Jo\v zef Stefan Institute\\
Ljubljana, Slovenia\\
\texttt{primoz.skraba@ijs.si} \\
}
\end{comment}
\author{
Jan Rupnik, Andrej Muhi\v c, Primo\v z \v Skraba \\
A.I. Laboratory\\
Jo\v zef Stefan Institute\\
Ljubljana, Slovenia\\
\texttt{(jan.rupnik),(andrej.muhic),(primoz.skraba)@ijs.si} \\
}
% The \author macro works with any number of authors. There are two commands
% used to separate the names and addresses of multiple authors: \And and \AND.
%
% Using \And between authors leaves it to \LaTeX{} to determine where to break
% the lines. Using \AND forces a linebreak at that point. So, if \LaTeX{}
% puts 3 of 4 authors names on the first line, and the last on the second
% line, try using \AND instead of \And before the third author name.

\newcommand{\fix}{\marginpar{FIX}}
\newcommand{\new}{\marginpar{NEW}}
\newcommand{\X}{\mathbb{X}}
\newcommand{\Y}{\mathbb{Y}}


\nipsfinalcopy % Uncomment for camera-ready version

\begin{document}


\maketitle

\begin{abstract}
We address the problem of learning similarities between documents written in different languages for language pairs where little or no direct supervision (in the form of a comparable or parallel corpus) is available. To make up for the lack of direct supervision, our approach takes advantage of the fact that they may be linked indirectly by a hub language. That is, correspondences exist between each of the languages and a third,  hub language.
The main goal of our paper is to explore the viability of cross-lingual learning under such conditions.
We propose a method that extracts a set of multilingual topics that facilitate a common representation of documents in different languages. The method is suitable for a comparable multilingual corpus with missing documents.  We evaluate the approach in a truly multi-lingual setting, performing document retrieval across eight Wikipedia languages.
\end{abstract}



\section{Introduction}
%
%Motivation: limited linguistic resources for several pairs, lack of tools to support machine learning tasks.
%Wikipedia comparable corpus can be used to leverage several machine learning tasks. Problem occurs for language pairs when direct document alignments are rare.


Document retrieval is a well-established problem in data
mining. There have been a large number of different approaches
put forward in the literature. In this paper we concentrate on a
specific setting: multi-lingual corpora. As the availability of
multi-lingual documents has exploded in the last few years, there is a pressing
need for automatic cross-lingual processing tools.
%
The prime example is Wikipedia - in 2001 the majority of pages
were written in English, while in 2012, the percentage of English
articles has dropped to 14\%. In this context, we look at how to
find similar documents across languages. In particular, we do not
assume the availability of machine translators, but rather try to
frame the problem such that we can use well-established machine
learning tools designed for monolingual text-mining tasks.

This work represents the continuation of previous work \cite{nips, iti, nips2, sikdd}
where we explored representations of documents which were valid
over multiple languages.  The representations could be interpreted
as multi-lingual topics, which were then used as proxies to compute
cross-lingual similarities between documents.
%
We look at a specific aspect of this problem: the distribution of
articles across languages in Wikipedia is not uniform. While the
percentage of English articles has fallen, English is still the
largest language and one of \emph{hub} languages which not only
have an order of magnitude more articles than other languages,
but also many comparable articles with most of the other languages.

When doing document retrieval, we encounter two quite different situations. If for example, we are looking for a
German article comparable to an English article, there is a large
alignment between the document corpora (given by the intersection
in articles in the two languages) making the problem
well-posed. If, however, we look for a relevant
Slovenian article given a Hindi article, the intersection is small,
making the problem much harder.
%However,
Since almost all languages have a large intersection in
articles with \emph{hub} languages, the question we ask in
this paper is: can we exploit hub languages to perform better
document retrieval between non-hub languages? A positive answer would vastly improve cross-lingual analysis between
less represented languages. In the following section, we
introduce our representation followed by our experiments, which
also shed light on the structural properties of the multilingual
Wikipedia corpus.

\section{Approach}
We begin by introducing some notation. A \emph{multilingual document} $d = (u_1,\ldots u_m)$ is a collection of $m$ documents on the same topic (comparable), where $u_i$ is the document in language $i$ which can be an empty document (missing resource) and $d$ must contain at least two nonempty documents. A comparable corpus $D = {d_1, \ldots, d_N}$ is a collection of multilingual documents. By using the standard vector space model (bag of words) we can represent $D$ as a set of $m$ matrices $X_1,\ldots,X_m$, where $X_i \in \RR^{n_i \times N}$ is the corpus matrix corresponding to the $i$-th language, where $n_i$ is the vocabulary size. Let $X_i^{\ell}$ denote the $\ell$-th column of matrix $X_i$. A \emph{hub language} is a language with a high proportion of non-empty documents in $D$, which in case of Wikipedia is English. Let $a(i,j)$ denote the index set of multilingual documents $d$ that contain non-empty $u_i$ and $u_j$ and $a(i)$ denote the index set of multilingual documents that contain $u_i$.

The goal of our approach is to find a language independent representation of documents by finding a set of mappings $P_1: \RR^{n_1} \rightarrow \RR^k ,\ldots, P_m: \RR^{n_m} \rightarrow \RR^k$ that map documents to a common $k$-dimensional vector space, where similarity between $P_i(x)$ and $P_j(y)$ reflects language independent similarity between $x$ and $y$.

The first step in our method is to project $X_1, \ldots, X_m$ to lower dimensional spaces without destroying the cross-lingual structure. Treating the columns of $X_i$ as observation vectors sampled from an underlying distribution $\mathcal{X}_i \in V_i = \RR^{n_i}$, we can analyze the empirical cross-covariance matrices: $C_{i,j} = \frac{1}{|a(i,j)|-1 }\sum_{\ell \in a(i,j)} (X_i^{\ell} - c_i)\cdot (X_j^{\ell} - c_j)^T$, where $c_i = \frac{1}{a_i} \sum_{\ell \in a(i)}X_i^{\ell}$. By finding low rank approximations of $C_{i,j}$ we can identify the subspaces of $V_i$ and $V_j$ that are relevant for extracting linear patterns between $\mathcal{X}_i$ and $\mathcal{X}_j$. Let $X_1$ represent the hub language corpus matrix. The standard approach to finding the subspaces is to perform the singular value decomposition on the full $N \times N$ covariance matrix composed of blocks $C_{i,j}$. If $|a(i,j)|$ is small for many language pairs (as it is in the case of Wikipedia), then many empirical estimates $C_{i,j}$ are unreliable, which can result in overfitting. For this reason we perform the truncated singular value decomposition on the matrix $C = [C_{1,2}  \cdots  C_{1,m}] \approx U S V^T$, where where $U \in \RR^{n_1 \times k}, S \in \RR^{k \times k}, V \in \RR^{(\sum_{i=2}^m n_i) \times k}$. We split the matrix $V$ vertically in blocks with $n_2, \ldots, n_M$ rows to obtain a set of matrices $V_i$: $V = [V_2^T  \cdots  V_m^T]^T$. Note that columns of $U$ are orthogonal but columns in each $V_i$ are not (columns of V are orthogonal). Let $V_1 := U$. We proceed by reducing the dimensionality of each $X_i$ by setting: $Y_i = V_i^T \cdot X_i$, where $Y_i \in \RR^{k\times N}$. The step is similar to Cross-lingual Latent Semantic Indexing (CL-LSI) \cite{lsi}\cite{cl_lsi} which is less suitable due to a large amount of missing documents. Since singular value decomposition with missing values is a challenging problem and the alternative of replacing missing documents with zero vectors can result in degradation of performance.

The second step involves solving a generalized version of canonical correlation analysis on the matrices $Y_i$ in order to find the mappings $P_i$. The approach is based on the sum of squares of correlations formulation by Kettenring \cite{Kettenring}, where we consider only correlations between pairs $(Y_1, Y_i), i >1$ due to the hub language problem characteristic.
 Let $D_{i,i} \in \RR^{k \times k}$ denote the empirical covariance and $D_{i,j}$ denote the empirical cross-covariance computed based on $Y_i$ and $Y_j$. We solve the following optimization problem:
\begin{equation*}%\label{eq:projZkdimapproxdualqcqp}
\begin{aligned}
& \underset{w_i \in \RR^{k}}{\text{maximize}}
& & \sum_{i = 2}^m  \left(w_1^T D_{1,i} w_i \right)^2
& \text{subject to}
& & w_i^T D_{i,i} w_i = 1, \quad\forall i = 1,\ldots, m.
\end{aligned}
\end{equation*}
By using Lagrangian multiplier techniques (we omit the derivation due to space constraints), the problem can be reformulated as the eigenvalue problem:$\left(\sum_{i=2}^m G_i G_i^T\right) \cdot V = \Lambda \cdot V,$
%\begin{equation*}
% \left(\sum_{i=2}^m G_i G_i^T\right) \cdot V = \Lambda \cdot V,
%\end{equation*}
 where $G_i = H_1^T D_{1,i} H_i$ and $H_i = Chol(D_{i,i})^{-1}$ where $Chol(\cdot)$ is the Cholesky decomposition: $X = Chol(X)^T Chol(X)$. The vectors $w_i$ can be reconstructed from the dominant eigenvector $v$, as: $w_1 = H_1 v$  and $w_i \propto H_i  G_i^T  v$ for $i >1$ ($w_i$ need to be appropriately normalized). Higher dimensional mappings $W_i$ are obtained in a similar way by setting the constraint $W_i^T D_{i,i} W_i = I$, where $I$ is the identity matrix. The constraint forces the columns of $W_i$ to be uncorrelated (orthogonal with respect to covariance matrix). $W_i$ can be extracted from a set of dominant eigenvectors of matrix $V$ according to eigenvalues $\Lambda$.
The technique is related to a generalization of canonical correlation analysis (GCCA) by Carroll\cite{Carroll}, where an unknown group configuration variable is defined and objective is to maximize the sum of squared correlation between the group variable and the others. The problem can be reformulated as an eigenvalue problem. The difference lies in the fact that we set the unknown group configuration variable as the hub language, which simplifies the solution. The complexity of our method is $O(k^3)$, whereas solving the GCCA method scales as $O(N^3)$ where $N$ is the number of samples (see \cite{gifi}). Another issue with GCCA is that it cannot be directly applied to the case of missing documents.

To summarize, we first reduced the dimensionality of our data to $k$-dimensional features and then found a new representation (via linear transformation) that maximizes directions of linear dependence between the languages. The final projections that enable mappings to a common space are defined as: $P_i(x) = W_i^T V_i^T x.$

\section{Experiments}
In this section we evaluate our proposed approach to learn the common document representations. The quality of the learned representation is evaluated on the task of cross-lingual document retrieval on Wikipedia. We will first describe the data and then present the evaluation.
%\subsection{Data}

To investigate the empirical performance of our algorithm %we will test it on a large-scale,
%real-world multilingual dataset that we extracted from Wikipedia  by  using so called 'inter-language' links as an alignment.
%Wikipedia is a large source of multilingual data that enables us to compare documents across the language barrier even for the minority languages that have
% no translation tools, multilingual dictionaries as Eurovoc, or strongly aligned multilingual
% corpora as Europarl available. %'en'    'es'    'ru'    'sl'    'hi'    'war'    'ht'    'pms'
we select a subset of Wikipedia languages containing three major languages, English--\emph{en} (hub language), Spanish--\emph{es}, Russian--\emph{ru}, and five minority (in the sense of Wikipedia sizes) languages, Slovenian--\emph{sl}, Piedmontese--\emph{pms}, Waray-Waray--\emph{war} %(all with about 2 million native speakers)
, Creole--\emph{ht}% (8 million native speakers)
, and Hindi--\emph{hi}.  %(180 million native speakers).
For preprocessing we remove the documents that contain less than 20 different words (stubs) and remove words occur in less than 50 documents as well as the top 100 most frequent words (in each language separately). We represent the documents as normalized TFIDF\cite{Salton88term-weightingapproaches} weighted vectors.
%  These are languages with Wikipedia article sizes comparable to the minority languages.  (since the number of speakers does not directly correlate with the number of Wikipedia articles).
%Some of selected Wikipedia's have  little alignment information available and are therefore good candidates for testing our approach to retrieval using hubs for when no direct alignment information is available. The prime hub candidate is the English language which is well aligned with all other Wikipedia languages, although the alignment quality varies quite a bit.
%Furthermore, we remove the documents that contain less than 20 different words and remove words that are too infrequent as well as the top 100 most frequent words in the vocabularies.
%Furthermore, we call the document consisting of less than 20 different words, a stub. This documents are typically garbage, the titles of the columns in the table, remains of the parsing process, or Wikipedia articles with very little or no information contained in one or two sentences.
% by using the standard vector space model where each term corresponds to a word or a phrase in a fixed vocabulary. An aligned corpus is represented by a set of matrices $D^J$ defined as $D^J := [d_1^J , \ldots , d_{m}^J ] \in \RR^{n_J \times m}$ where $m$ is the number of documents in the aligned corpus and $n_J$ is the dictionary size. Let $N = \sum_{J=1}^M n_J$, where $M$ is the number of languages. The multi-lingual document $k$ of the aligned corpus is formed as a block vector
%$d_k = [ {d_k^1}^T \dots {d_k^M}^T]^T \in \RR^{N}$, and documents are indexed so that $d_k^{J_r}$ and $d_k^{J_s}$ is an aligned document pair for all $k, J_r, J_s,$
%where $1\leq r \leq M,$ $1\leq s \leq M,$ and $r \neq s.$ If the alignment information for the multi-lingual document $k$ in language $j$ is missing, then $d_k^j = 0.$
%additionally identified by its corresponding language.
%Each language has also its own dictionary independent of other languages.
%The individual
%term-document matrices are grouped according to language giving $
%\{ D_1, D_2, \ldots, D_\ell\}$.
%Our main interest are similarities between documents. We can measure the similarities between documents of the same language
% by using the cosine similarity. We note that given a
%transformation function between dictionaries, we could compute
%similarities between documents in different languages. However,
%bilingual-dictionaries are not available for many language pairs in the Wikipedia corpus
%and we will rely on document correspondences to compute document similarities across
%languages in the present paper. Note that this implies a correspondence between
%columns in each $D^J$ for all $J$. Translations between
%dictionaries would give us row-based correspondences, but we will
%investigate this approach in future work.
%As the first step of the processing, we improve the quality of the alignment by removing the stub documents.
%This greatly improves the chance of the successful cross-lingual information retrieval.
%As terms in the dictionary are not equally important for
%determining similarity between documents, we preprocess
%the documents.
%First, we remove the infrequent words and the top 100 most frequent words in every language.
%We form pairs ($\ell$,~$n$) such that a word is infrequent in language $\ell$ if it appears in too few documents $n,$ (\emph{en},~50), (\emph{es},~40), (\emph{ru},~40), (\emph{sl},~10), (\emph{hi},~5), (\emph{war},~5), (\emph{ht},~5), (\emph{pms},~5).
%The criteria for the infrequency was chosen depending on the size of the corpora and in such a way that enough vocabulary was retained for smaller sized Wikipedias.
%At the end we additonally  use term frequency ($TF$), to prune away
%infrequent terms.
%Rather than take a fixed number of top terms in
%each document we use an adaptive measure.
%Let $f(n)$ be a map
%which returns numbers of terms appearing at least $n$ times.  For each language $k$
%we find the maximal $n$ such that
%$\max(10^5, N_J)\leq f(n) \leq 6\cdot 10^5,$ where $N_J$ is number of documents in complete
%dictionary of language $J.$ As we do not apply any stemming and lemmatization this ensures that we get well balanced corpus and retain at least of
%$10^5$ words for languages with rather small Wikipedias but potentially rich vocabularies.
%find $n$ such that
%\begin{equation}
%\frac{|f(n+1) - f(n)|}{\mbox{original \# of terms}} < 0.001
%\end{equation}
%and used these as the terms for the document.
%Once this pruning step is complete, we represent the documents as normalized TFIDF\cite{Salton88term-weightingapproaches} weighted vectors.
%step is complete, we further re-weight the remaining terms.
%Our data is a collection of documents in multiple languages along
%with cross-lingual alignment information. Each document is represented as a bag of words vector consisting of term frequencies (TF).
%A term
%weight should correspond to the importance of the term for the
%given corpus.  The common weighting scheme is called Term
%Frequency Inverse Document Frequency (TFIDF) weighting. An
%Inverse Document Frequency (IDF) weight for the dictionary term $j$
%is defined as $w_j = \log( M /DF_j )$, where $DF_j$ is the number of
%documents in the $M$ document corpus which contain term $j$.  A document
%TFIDF vector is its original vector multiplied element-wise by the
%weights. The $j$-th element of a document vector is given by $
%TF_j \log( M/DF_j )$.
%
%Furthermore, we re-normalize each vector to have
%Euclidean norm equal to 1. This improves the balance in the aligned data and helps to partially resolve the problem when one very short document in one language is
%aligned with large document in other language.
%Finally, we prepare the data for the experiments. We will test retrieval using English language as a hub. Therefore, we keep only the documents that appear in at least one English--(other language) pair.
%metanet
%\subsection{Evaluation}

The evaluation is based on splitting the data into training and test sets (described later). On the training set, we perform the two step procedure to obtain the common document representation as a set of mappings $P_i$. A test set for each language pair, $test_{i,j} = \{(x_\ell,y_\ell) | \ell = 1:n(i,j)\} $, consists of comparable document pairs (linked Wikipedia pages), where $n(i,j)$ is the test set size. We evaluate the representation by measuring mate retrieval quality on the test sets: for each $\ell$, we rank the projected documents $P_j(y_1),\ldots, P_j(y_{n(i,j)})$ according to their similarity with $P_i(x_\ell)$ and compute the rank of the mate document $r(\ell) = rank(P_j(y_\ell))$. The final retrieval score (between -100 and 100) is computed as: $\frac{100}{n(i,j)} \cdot \sum_{\ell = 1}^{n(i,j)} \left( \frac{n(i,j) - r(\ell)}{n(i,j) -1} -0.5\right)$. A score that is less than 0 means that the method performs worse than random retrieval and a score of 100 indicates perfect mate retrieval. The mate retrieval results are included in Table \ref{table:retrieval}.

We observe that the method performs well on all pairs between languages: \emph{en}, \emph{es}, \emph{ru}, \emph{sl}, where at least 50,000 training documents are available. We notice that taking $k = 500$ or $k = 1000$ multilingual topics usually results in similar performance, with some notable exceptions: in the case of (\emph{ht}, \emph{war}) the additional topics result in an increase in performance, as opposed to (\emph{ht},\emph{pms}) where performance drops, which suggests overfitting. The languages where the method performs poorly are \emph{ht} and \emph{war}, which can be explained by the quality of data (see Table \ref{table:rank} and explanation that follows). In case of \emph{pms}, we demonstrate that solid performance can be achieved for language pairs (\emph{pms}, \emph{sl}) and (\emph{pms}, \emph{hi}), where only 2000 training documents are shared between \emph{pms} and \emph{sl} and no training documents are available between \emph{pms} and \emph{hi}. Also observe that in the case of (\emph{pms}, \emph{ht}) the method still obtains a score of 62, even though training set intersection is zero and \emph{ht} data is corrupted, which we will show in the next paragraph.
{
\renewcommand\tabcolsep{3pt}
\begin{table}[h!]
\caption{Pairwise retrieval, 500 topics;1000 topics}\label{table:retrieval}
\begin{center}
\begin{tabular}{|c|c|c|c|c|c|c|c|c|}
\cline{1-9}
&	en&	es&	ru&	sl&	hi&	war&	ht&	pms\\\cline{1-9}
en&	    &	98~-~98&	95~-~97&	97~-~98&	82~-~84&	76~-~74&	53~-~55&	96~-~97\\
\cline{1-9}
es&	97~-~98&	&	94~-~96&	97~-~98&	85~-~84&	76~-~77&	56~-~57&	96~-~96\\
\cline{1-9}
ru&	96~-~97&	94~-~95&	&	97~-~97&	81~-~82&	73~-~74&	55~-~56&	96~-~96\\
\cline{1-9}
sl&	96~-~97&	95~-~95&	95~-~95&	&	91~-~91&	68~-~68&	59~-~69&	93~-~93\\
\cline{1-9}
hi&	81~-~82&	82~-~81&	80~-~80&	91~-~91&	&	68~-~67&	50~-~55&	87~-~86\\
\cline{1-9}
war&	68~-~63&	71~-~68&	72~-~71&	68~-~68&	66~-~62&	&	28~-~48&	24~-~21\\
\cline{1-9}
ht&	52~-~58&	63~-~66&	66~-~62&	61~-~71&	44~-~55&	16~-~50&	&	62~-~49\\
\cline{1-9}
pms&	95~-~96&	96~-~96&	94~-~94&	93~-~93&	85~-~85&	23~-~26&	66~-~54&	\\
\cline{1-9}
\end{tabular}
\end{center}
\end{table}
}
We now describe the selection of train and test sets. We select the test set documents as all multi-lingual documents with at least one nonempty alignment from the list: (\emph{hi}, \emph{ht}), (\emph{hi}, \emph{pms}), (\emph{war}, \emph{ht}), (\emph{war}, \emph{pms}). This guarantees that we cover all the languages. Moreover this test set is suitable for testing the retrieval thorough the hub as
the chosen pairs have empty alignments.
The remaining documents are used for training.
In Table \ref{table:train_test}, we display the corresponding sizes of training and test documents for each language pair.
The first row represents the size of the training sets used to construct the mappings in low dimensional language independent space using the English--\emph{en} as a hub.
The diagonal elements represent number of the unique training documents and test documents in each language.
{
%\renewcommand{\arraystretch}{0.5}
\renewcommand\tabcolsep{3pt}
\begin{table}[h!]
\caption{Pairwise training:test sizes (in thousands)}
\label{table:train_test}
{
\small
\begin{tabular}{c|c|c|c|c|c|c|c|c|}
&	en&	es&	ru&	sl&	hi&	war&	ht&	pms\\\cline{1-9}
en&	671~-~4.64&	463~-~4.29&	369~-~3.19&	50.3~-~2&	14.4~-~2.76&	8.58~-~2.41&	17~-~2.32&	16.6~-~2.67\\
\cline{2-9}
es&	\multicolumn{1}{c|}{}	&	463~-~4.29&	187~-~2.94&	28.2~-~1.96&	8.72~-~2.48&	6.88~-~2.4&	13.2~-~2&	 13.8~-~2.58\\
\cline{3-9}
ru&	\multicolumn{2}{c|}{}	&	369~-~3.19&	29.6~-~1.92&	9.16~-~2.68&	2.92~-~1.1&	3.23~-~2.2&	10.2~-~1.29\\
\cline{4-9}
sl&	\multicolumn{3}{c|}{}	&	50.3~-~2&	3.83~-~1.65&	1.23~-~0.986&	0.949~-~1.23&	1.85~-~0.988\\
\cline{5-9}
hi&	\multicolumn{4}{c|}{}	&	14.4~-~2.76&	0.579~-~0.76&	0.0~-~2.08&	0.0~-~0.796\\
\cline{6-9}
war&	\multicolumn{5}{c|}{}	&	8.58~-~2.41&	0.043~-~0.534&	0.0~-~1.97\\
\cline{7-9}
ht&	\multicolumn{6}{c|}{}	&	17~-~2.32&	0.0~-~0.355\\
\cline{8-9}
pms&	\multicolumn{7}{c|}{}	&	16.6~-~2.67\\
\cline{9-9}
\end{tabular}
}
\end{table}
}

We further inspect the properties of the training sets by roughly estimating the fraction \texttt{rank(A)/min(size(A))} for each training English matrix and its corresponding mate matrix.
Ideally, these two fractions are approximately the same so  both aligned spaces should have reasonably similar
dimensionality.
We display these numbers as pairs in Table \ref{table:rank}.
\begin{table}[h]
\caption{Dimensionality drift}
\label{table:rank}
\begin{tabular}{|c|c|c|c|c|c|c|}
\cline{1-7}
(en, de)     &   (en, ru)     &   (en, sl)       &     (en, hi)&   (en, war)      &      (en, ht) &   (en, pms)\\
\cline{1-7}
(0.81, 0.89)   &  (0.8, 0.89)  &   (0.98, 0.96)    &    (1, 1)  &   (0.74, 0.56)  &      (1, 0.22)  &   (0.89, 0.38)\\
\cline{1-7}
\end{tabular}
\end{table}
It is clear that in the case of Creole language only at most $22\%$ documents are unique and suitable for the training.
Though we removed the stub documents, many of remaining documents are almost the same, as the quality of some minor Wikipedias is low. This was confirmed for Creole, Waray-Waray, and Piedmontese language
by manual inspection. The low quality documents correspond to templates about the year, person, town, etc. and contain very few unique words.


We also have a problem with the quality of the test data. For example, if we look at test pair (\emph{war}, \emph{ht}) only 386/534 Waray-Waray test documents are unique but on other side
almost all Creole test documents (523/534) are unique. This indicates a poor alignment which leads to poor performance.





\section{Conclusions}
We proposed a method that enables finding common representations for documents in different languages that is tailored
to minority language pairs with limited direct linguistic resources (rare or no comparable document pairs, no dictionary information). We demonstrated that the discovery of cross-lingual mappings is possible even if a pair of languages has no shared linguistic resources, provided that they both share some document correspondences with a hub language.

\section{Acknowledgements}
The authors gratefully acknowledge that the funding for this work was provided by the projects X-LIKE (ICT-257790-STREP)\cite{xlike} and META-NET (ICT-249119-NoE)\cite{metanet}.
%The authors gratefully acknowledge that the funding for this work was provided by the projects X-LIKE (ICT-257790-STREP)\cite{xlike},  MultilingualWeb (PSP-2009.5.2 Agr.\# 250500)\cite{mlweb}, TransLectures (FP7-ICT-2011-7)\cite{translectures}, PlanetData (ICT-257641-NoE)\cite{pdata}, RENDER (ICT-257790-STREP)\cite{render}, and META-NET (ICT-249119-NoE)\cite{metanet}.

\bibliographystyle{unsrt} \bibliography{nips_xlite}
%\bibliographystyle{plain}\bibliography{nips_xlite}


\end{document}
